Physics, as a discipline in the modern world, is fundamentally concerned with understanding the parts of reality that are unknown. Much of modern research can be broadly framed within three frontiers, each defined by what knowledge remains elusive. 
One frontier is the very large; the scale of astrophysics and cosmology, characterised by measures of size, energy, and duration that are larger than the human scale. We build telescopes and detectors to search for evidence for theoretical models. 
Another frontier is the very small; the scale of quantum fields and fundamental particles, characterised by measures of size, energy, and duration that are smaller than the human scale. We build colliders, resonant cavities, and precise experiments to search for evidence for theoretical models. 

An emerging frontier of physics research is the intractable; systems whose behaviour is not characterised by distance or scale, but by overwhelming complexity. Many physical systems well-described by the tools of physics remain unsolved, such as the behaviour of many-body quantum systems in materials, chemistry, and biology, or the solutions to large combinatorial optimisation problems in logistics, machine learning, and networks. Addressing problems in this domain requires new approaches: we design new algorithms and software and build quantum computers to push beyond classical computational limits. This is the context that motivates the development of quantum computing.

Quantum computing is a rapidly evolving field that leverages the principles of quantum mechanics to address problems that are intractable for classical computation. By utilising the unique properties of quantum bits (qubits), such as superposition and entanglement, quantum computers can perform computational tasks that classical systems cannot mimic. Realising the potential of this technology requires specialised quantum algorithms that restructure how problems are encoded and solved. Advances in hardware, algorithms, and applications are driving a growing interest in quantum computing across a wide range of scientific and technological domains.

One domain where quantum algorithms may offer significant advantages is graph theory.  Graphs are fundamental models for complex systems and underpin many applications, such as search engines, social networks, logistics, transportation, and communication. In many such contexts, it is essential to quantify the similarity between graphs.
For example, detecting changes in interaction patterns over time, analysing structural differences between graphs.\todo{finish this and cite}
In all these cases, the similarity between two graphs with similar sets of nodes is assessed, and the detection of changes in the patterns of connectivity is important. For graphs that may match approximately, it is useful to obtain a measure of similarity.

Graph similarity measures are quantitative calculations based on comparisons made between the structure of network graphs. Different measures of graph similarity will produce a variety of results because of differences in how the structures of the graphs are analysed. There are many measures of graph similarity, including Maximum Common Subgraph, Graph Edit Distance, Frobenius Distance, and Edge Overlap.\todo{cite these} These measures are successful in indicating the degree to which two graphs are similar, and are used in practical tasks involving graph analysis.

However, graph similarity measures require a vertex-to-vertex matching between the two graphs, which is computationally expensive to determine. In particular, determining the optimal matching between two unlabelled graphs is an NP-hard problem, meaning that there is no known algorithm can exactly solve all instances in polynomial-time. As a result, various heuristic and approximation techniques have been developed to tackle the graph similarity problem.

A recently introduced quantum algorithm, the non-variational quantum walk-based optimisation algorithm (NVQWOA), was designed to solve hard combinatorial optimisation problems \cite{bennett2024nonvariational,bennett2024analysisnonvariationalquantumwalkbased}. NVQWOA amplifies the probability of measuring high-quality solutions to a problem through an iterative process of phase-separation and mixing via a continuous-time quantum walk (CTQW) on a graph associated with the problem's structure. In the original work, it was shown through classical simulation that NVQWOA found globally optimal solutions on problems such as weighted maxcut, $k$-means clustering, quadratic assignment, and the capacitated facility location problem within a small number of iterations.

We seek to determine whether the NVQWOA will perform equally well on the problem of calculating graph similarity as it performed on its first problems. We approach this through classical simulation of the NVQWOA on a range of instances of the graph similarity problem, analysing its behaviour, scalability, and potential as a quantum algorithm for this important class of problems.
