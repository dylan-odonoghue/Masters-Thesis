Quantum computing is a rapidly evolving field that leverages the principles of quantum mechanics to perform computational tasks. By utilising the unique properties of quantum bits (qubits), such as superposition and entanglement, quantum computers can solve certain problems more efficiently than classical computers. This has led to significant interest in exploring quantum algorithms and their applications across various domains.

Network graph models have become crucial components in systems that are used in everyday life, such as search engines and social networks. Graph similarity is often important in these contexts. For example, social networks are compared to identify certain interaction patterns, traffic networks are compared to help detect abnormal changes in traffic patterns, and web networks are compared for anomaly detection. In all these cases, the similarity between two graphs with overlapping sets of nodes is assessed, and the detection of changes in the patterns of connectivity is important. For graphs that match approximately, it is useful to obtain a quantitative measure of similarity.

Graph similarity measures are quantitative calculations based on comparisons made between the structure of network graphs. Different measures of graph similarity will produce a variety of results because of differences in how the structures of the graphs are analysed. There are many measures of graph similarity, including Maximum Common Subgraph, Graph Edit Distance, Frobenius Distance, and Graph Eigendecomposition. All of these measures are successful in indicating the degree to which two graphs are similar and can also detect small differences between them.

However, many of these measures require a matching between the vertices of the two graphs, which can be computationally expensive to determine. The problem of finding an optimal vertex matching is known to be NP-hard, meaning that there is no known polynomial-time algorithm to solve it for all instances. As a result, various heuristic and approximation techniques have been developed to tackle the graph similarity problem.

A recently introduced algorithm, the non-variational QWOA was designed to solve hard combinatorial optimisation problems, generalising to non-binary and constrained problems, while simultaneously solving the challenges related to the variational approach \cite{bennett2024nonvariational}. It amplifies the probability of measuring high-quality solutions to a problem through a process of phase-separation and mixing via a continuous-time quantum walk (CTQW) on a mixing graph. In the original work, it was shown through classical simulation that non-variational QWOA found globally optimal solutions on problems such as weighted maxcut, $k$-means clustering, quadratic assignment, and the capacitated facility location problem within a small number of iterations.

We seek to determine whether the non-variational QWOA will perform equally well on the problem of calculating graph similarity as it performed on its first problems. We will do this through classical simulation of the non-variational QWOA on small instances of the graph similarity problem. We will compare its performance to that of Grover's search algorithm \cite{grover_search}, which is a well-known quantum algorithm for unstructured search problems. Grover's algorithm provides a quadratic speedup over classical search algorithms, making it a useful benchmark for evaluating the performance of other quantum algorithms.
\\
In this report, we will first provide a brief overview of the non-variational QWOA and Grover's search algorithm. We will then describe our methodology for simulating these algorithms on small instances of the graph similarity problem. Finally, we will present our results and discuss their implications for the use of quantum algorithms in solving combinatorial optimisation problems.
