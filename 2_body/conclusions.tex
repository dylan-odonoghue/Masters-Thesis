This thesis analysed using the NVQWOA on the unlabelled graph similarity problem. Using classical simulation techniques, we characterised the performance of the algorithm and assessed its potential for use in graph similarity tasks.

Our results demonstrate that the NVQWOA consistently amplifies optimal and near-optimal solutions across a range of problem sizes. The amount of amplification applied to each solution closely tracks the solution's proximity to the optimum, quantified by either Hamming distance or subshell distance, and align with expected behaviour for the continuous-time quantum walk on the mixing graph. The stability of the hyperparameters suggest that these hyperparameters coyld be used in a parameter transfer to instances with a larger problem size $n$.

TBD.