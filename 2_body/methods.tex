\section{Overview}
We classically simulated the application of the NVQWOA to the graph similarity problem. The measure of graph similarity chosen was Edge Overlap, as it is a low-cost measure that still serves as a interpretable measure of similarity. The aim was to maximise this objective function through matrix exponentiation and multiplication,analytically computing the manipulation of state vector such that the states which corresponded with the optimal solutions (those with the greatest similarity) had an amplified probability of being measured.

\section{Classical Simulation}
To investigate the performance of the NVQWOA on graph similarity problems, we chose to perform classical simulations of a quantum computer, rather than directly performing the algorithm on a quantum computer. Classical simulation allows for computation with negligible stochastic or decoherence errors and complete analysis of the evolving quantum state, something which is inaccessible on real quantum hardware but useful for characterising performance and benchmarking.

Classical simulation of quantum algorithms and computers can be approached in many different ways. Tensor networks simplify circuits into lower-entanglement systems but has exponentially degrading performance as circuit depth and entanglement entropy. Stabiliser-based approaches can efficiently simulate Clifford-only circuits and certain restricted non-Clifford extensions but cannot capture the arbitrary parameterised unitaries used in the NVQWOA. State vector simulation computes the full evolution of the state vector, but can be costly in computational resources.\todo{citations}

State vector simulation was chosen as the strategy for its accuracy. The Qiskit Aer simulator and related packages were not suitable as the permutation mixer required is a sparse operator rather than a tensor-product of standard gates. Instead, we directly computed the state vector using QuOp, a Python framework for parallel simulation of quantum variational algorithms\cite{QuOp_MPI_paper_variational,QuOp_MPI}. QuOp provides efficient handling of large, sparse unitaries and supports MPI-based parallelisation, allowing simulation of high-dimensional state vectors across multiple compute nodes without loss of precision.

Classical simulations of the non-variational QWOA would not be possible without high-performance computing (HPC) resources, which were provided by the Pawsey Supercomputing Centre.

\section{QuOp}
QuOp is a Python 3 module designed for parallel, distributed-memory simulation of QVAs with arbitrary phase-shift and mixing operators. 

To apply it to the NVQWOA, a parameter mapping was used to translate the hyperparameters of the NVQWOA to the individual parameters of a QVA. The quantum circuit was defined to be the two unitaries of the NVQWOA, the phase separator $U_Q$ and the mixer $U_M$. With these modifications, QuOp could simulate the NVQWOA.

\section{Graph Similarity Measure}
As described in section \ref{graph_sim_section}, there are many options for the choice of which graph similarity measure to use.

Graph edit distance is the most common choice, but is costly to compute, as it requires the equivalent of a pathfinding search, such as an $A^*$ search algorithm.\todo{cite}
Since computational resources were a significant pressure, we chose one of the least computationally expensive similarity measures to compute, the edge overlap.

To calculate the edge overlap, the number of matching entries in the adjacency matrices of the graphs are counted and the sum is divided by $n^2$, which corresponds to the size of the adjacency matrices and the maximum possible overlap. This easy calculation also makes the edge overlap highly interpretable. One can imagine the vertex mapping as translating the vertices of one graph to the corresponding vertices in the other graph, and the edge overlap represents how much the edges line up correctly. The edge overlap also works on directed graphs, but directed graphs were not investigated in this thesis.

When the best vertex mapping is known between the two graphs, the edge overlap satisfies Koutra's similarity measure axioms, making it a `good' measure of graph similarity.

A drawback of using edge overlap is that it is not well-defined for graphs with differing numbers of nodes, $|V_1|\neq|V_2|$. This restricts our investigation to graphs with an equal number of nodes.

The choice of graph similarity measure is not expected to have a strong effect on the observed results. The vertex mappings that give strong similarity with any particular measure usually give strong similarity with most measures. For the choice of a specific graph similarity measure to affect the results, it would have to give a different problem structure which significantly alters the structure of qualities.

\section{Problem Instance Construction}
There does not exist a widely used library of graph similarity problem instances. To test the algorithm on a wide group of fairly-chosen problems, we randomly generated the graphs using the Erdős-Renyi random graph model.\todo{cite erdos renyi}
The number of nodes $n$, was varied to investigate how changing problem sizes changed the performance of the algorithm. The probability of edge inclusion was fixed at 0.5 for simplicity and repeatability.

Differences in graphs were induced by copying a graph and removing $k$ edges. This allowed the upper limit of similarity to be easily known to be $\frac{n^2-2k}{n^2}$.

\section{Phase Separator and Mixer Construction}
The phase separating unitary was constructed in QuOp. For each vertex mapping, the quality (given as edge overlap) was calculated and stored as an array. This array was converted into a diagonal unitary operator in QuOp.

The mixer was constructed as a column oriented array in Numpy and converted to a compressed sparse row format for storage. This sparse matrix could then be loaded into QuOp using HDF5 parallel I/O operations.\todo{cite HDF5?}

Since we are considering graphs with an equal number of nodes $n$, vertex mappings match the arbitrary vertex indices of each graph to one another with some ordering, $\pi:\{1,...,n\}\to\{1,...,n\}$. (In the unlabelled graph problem, the vertices in each graph can still have indices, just not information that can inform how to match those indices together.) Each $\pi\in\Pi$ corresponds to a distinct reordering of the vertex indices. Therefore, vertex mappings can be equivalently written as permutations of $n$ distinct decision variables.

From this, we can conclude that the correct choice of mixer is the permutation mixer, which connects solutions that are one swap of decision variables away from one another. In terms of vertex mappings, the permutation mixer connects solutions corresponding to mappings which differ only by two vertices that have had their targets swapped. Since these are the nearest neighbours to one another, we can also conclude that the minimum hamming distance from one solution to another is 2.

\section{Hyperparameter Optimisation} \label{sec:parameters}
The NVQWOA has three hyperparameters: $\gamma, t, \beta$. Although sub-optimal parameters can produce quantum circuits that significantly amplify optimal and near-optimal solutions, it is expected that finding optimal values for these hyperparameters through three-dimensional numerical optimisation will give the best possible results.\cite{bennett2024analysisnonvariationalquantumwalkbased}
 
We used three different objective functions as inputs to the optimiser:
\begin{enumerate}
    \item \textit{Expected Value}: The expected value of the state vector after amplification. This is the most practical to measure on real quantum hardware.
    \item \textit{Optimal Solution Probability (OSP)}: The probability of measuring an optimal solution. This is impractical to measure on quantum hardware, as it requires foreknowledge of the optimal solution and an unfeasible number of shots. It is included in our investigation here to provide an upper bound by finding the best possible parameters.
    \item \textit{Conditional Value at Risk (CVaR)}: The expected value of the $\alpha$-tail of the state vector. This takes more shots to measure on quantum hardware than the expected value, but has been shown to give faster convergence to better solutions.\cite{CVaR_quantum}
\end{enumerate}
Optimisation of the NVQWOA hyperparameters was done in QuOp during the NVQWOA simulation, using the Nelder-Mead method of numerical optimisation \cite{nelder_mead}. This method was chosen after experimentation with various numerical optimisation methods showed that Nelder-Mead converged the most quickly to the optimal parameters.

\section{Procedure}
\subsection{Amplification Analysis}
We started the analysis of the NVQWOA by considering one pair of graphs at a time, and verifying that amplification is correctly applied to the optimal and near-optimal solutions.

For each of $n=8,9,10,$ the probability distribution for the different solutions was compared before and after the algorithm was applied. We expected that solutions with high similarity would be more likely to be measured after the algorithm was applied. Additionally, the probability to measure the optimal solution was calculated.

We later expanded this analysis to problem sets with 30 distinct problem instances.

\subsection{Solution Distance Analysis}
To supplement the verification of amplification, it is useful to obtain statistics on the distance between solutions. This is because we expect that near-optimal solutions, which will have a low distance to the optimal solutions, should be amplified more than other solutions, which have a greater distance to the optimal solutions.

There are two useful notions of distance between solutions: subshell distance and Hamming distance, which are defined and explained below.

\subsubsection{Subshell Distance}
In this thesis, we use the term subshell distance to refer to the minimum distance on the mixing graph between a solution and an optimal solution. The term subshell in this notion of distance comes from the description in section \ref{sec:subshell}, and is used to distinguish this distance measure from other approaches. The mixer we are using is the permutation mixer, so the subshell distance is defined as:
$$\text{SD}(\mathbf{x}^1,\mathbf{x}^2)=\min \{m\in \mathbb{N}_0: \exists \tau_1,...,\tau_m \in \mathcal{T} \text{ with } \tau_m ... \tau_1 \mathbf{x}^2 = \mathbf{x}^1\}$$
where $\mathcal{T}=\{(ij):1 \leq i \leq j \leq n \}$ is the set of all transpositions in the symmetric group $S_n$ and transpositions act by reindexing coordinates. For this analysis $\mathbf{x}^1$ is always an optimal solution. If there were multiple optimal solutions, the subshell distance from each optimal solution was found and the minimum value was used.

Subshell distance is a useful metric in the analysis because it directly represents how close the solutions are to each other on the mixing graph. By analysing how amplification is applied to solutions depending on their subshell distance, we can observe how probability is being moved by the mixing process. Additionally, if one were to perform a local search after finding a near-optimal solution, it would likely be a search of solutions with low subshell distance to the found solution, to limit the search space.

\subsubsection{Hamming Distance}
Hamming distance is a simple yet powerful and highly interpretable measure for comparing objects. It measures the number of different decision variables for a solution. The Hamming distance for two solutions is defined as 
$$\text{HD}(\mathbf{x}^1,\mathbf{x}^2)=\sum_{i=0}^n \delta(x^1_i,x^2_i)$$
where $\delta$ is 1 if $x^1_i\neq x^2_i$ and 0 otherwise. For this analysis $\mathbf{x}^1$ is always an optimal solution. If there were multiple optimal solutions, the Hamming distance from each optimal solution was found and the minimum value was used.

Hamming distance is a useful metric in the analysis because it gives a measure of distance that is not related to the structure of the algorithm and can be easily compared to other combinatorial optimisation problems and algorithms.

\subsection{Mixer Analysis}
Using the notion of subshell distance, we can characterise the suitability of this mixer to the problem of graph similarity. 


\subsection{Hyperparameter Analysis}
The hyperparameters were selected by the optimisation procedure outlined in section \ref{sec:parameters}, beginning with the initial parameters $\gamma=1, t=0.2, \beta=0.8$. These were chosen as the initial parameters by rounding an initial run of numerical optimised parameters to one significant figure.

We stored the hyperparameters obtained for each problem instance with each optimisation objective, producing a set of nine hyperparameters for each problem instance.