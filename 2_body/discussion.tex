The results of each area of investigation were largely expected and particularly show the effectiveness of using CVaR as a hyperparameter optimisation objective function.

\section{Amplification Verification}
The probability to measure the optimal and near-optimal solutions was consistently amplified by the NVQWOA, as shown in Figures \ref{fig:similarity dist}, \ref{fig:similarity log dist}, and \ref{fig:osp}.

\section{Hyperparameters}
The choice of hyperparameter optimisation objective function had a large impact on the amount of amplification applied to the optimal and near-optimal solutions. 

\section{Solution Distance}
Hamming distance is less smooth than subshell distance because there is degeneracy in the Hamming distance even for solutions with differing solution distance. For example, the permutation $[1,2,3,4]$ has a Hamming distance of 4 from both $[2,1,4,3]$ and $[2,3,4,1]$, but its subshell distances would be 2 and 3 respectively. Solutions with different subshell distances will be distributed differently on the mixing graph, so the probability will not be as concentrated.

\section{Mixer Analysis}\label{sec:mixer discussion}

Mean quality gap

Subshell quality variance